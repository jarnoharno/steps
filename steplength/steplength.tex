\documentclass[a4paper,parskip=half]{scrartcl}
\usepackage[english]{babel}
\usepackage[utf8]{inputenc}
\usepackage[T1]{fontenc}
\usepackage{mathptmx}
\usepackage[numbers]{natbib}
\usepackage{hyperref}
\usepackage{enumitem}
\usepackage{amsmath}
\setlist[description]{font={\rmfamily}}
\addtokomafont{disposition}{\rmfamily}

\author{Jarno Leppänen}
\title{Step period vs. step length}

\begin{document}

\maketitle

\section{Model}

A log-log regression equation describes the relationship between step frequency
and speed of walking better than linear equation for
adults.\cite{grieve1966relationships} This relationship can be described
mathematically as
\begin{align} \label{eq:fvb}
f \sim v^b,
\end{align}
where $f$ is the step frequency and $v$ is the walking speed. Bertram and Ruina
gives the empirically measured value $b = 0.58$ for the exponent while Kuo gives
$b = 0.42$.\cite{bertram2001multiple,kuo2001simple}

Given that $f = 1/s$ and $v = fd$, where $s$ is the step period (temporal length of a step)
and $d$ is the spatial step length, equation \ref{eq:fvb} can be written as
\begin{align} \label{eq:dsa}
d \sim s^a,
\end{align}
where $a = (b - 1)/b$.

\bibliographystyle{plainnat}
\bibliography{steplength}
\end{document}
